By replicating the experiments in Shin and Liu's essay\cite{shin2021neuronized}, it is found that there are still some differences between our experimental results for simulated and real data and the results given by the authors. However, while the same analysis program was used, the difference between the results for the real data is relatively small, while the difference between the results for the simulated data is large, which could indicate that the difference in the results may come from the difference between the simulated sample data set and the original work, while the data processing process itself is not too problematic.

From the results in the table, we can see that the neuronized prior usually has a smaller ESS than the original prior, which proves that the neuronizing operation can improve the efficiency of analyzing the data, and the difference is most obvious in horseshoe prior.

As in the Shin and Liu's essay, the MSE of lasso shows the best estimation performance, but the false positive is higher. By comparing MCC, we found that SpSL prior generally has a better model selection process in high dimensional settings, while horseshoe and its neuronized prior has a preferred result in low-dimensional setting.